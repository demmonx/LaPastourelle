\par Pour le schéma logique de données, les @ représentent les clés primaires, 
et \# les clés étrangères. \\
\begin{verbatim}
languages_world (@id, nom_en, locale, nom_fr);
continent (@cont_id, cont_code, cont_name);
countries (@id_pays, code, name_en, name_fr, #code_continent);
lang (@lang_id, #lang_code, lang_img);
page (@page_id, page_code, page_nom);
phrase_jour (@phrase_id, phrase_content, #phrase_lang);
actualite (@act_id, act_type, act_img, act_nom);
actu_content (id, @#lang, @#actu, content);
coordonnees (@coord_num, coord_adr, coord_mail, coord_tel, coord_lat, 
coord_long);
produits (@pd_num, pd_prix, pd_img, pd_nom_prive, pd_nom_admin);
produits_contenu (bt_num, @#bt_lang, @#bt_prod, bt_content, bt_nom_public);
tmembre_inscrit (@id_membre, pseudo, pass_secure, niveau, email, 
etat_validation, telephone, nom, prenom, adresse, etat_annuaire);
photo (@id_photo, adr_photo, date_photo, description);
commentaire (@id_commentaire, texte, #num_photo, #auteur);
diaporama (@diapo_id, diapo_lien, diapo_active);
uploaded_file (@file_num, file_adr);
lien_ext(@lien_num, lien_url, lien_img, lien_nom);
livreor (@id, date, nom, message, validation);
voyage (@id_voy, #pays, titre, texte);
planning (@id_planning, pl_jour, pl_date, pl_lieu, pl_musiciens);
playlist (@music_id, music_lien, music_nom, music_active, music_groupe);
revue_presse (@presse_num, presse_img, presse_titre);
texte (txt_num, @#txt_page, txt_titre, @#lang, texte);
compte_rendu (@cr_num, cr_text, cr_date);
titre (@titre_num, titre_nom, titre_link);
traduction (trad_num, @#code_lang, @#code_titre, content);
\end{verbatim}